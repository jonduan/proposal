\section{Data and work completed to
date}\label{data-and-work-completed-to-date}

According to the United Nations Environment Programme (UNEP et al.
2016), the investment in renewables excluding large hydro represented
about 24\% of the all new capacity electrical generating capacity
installed globally in 2015, which is the first-time renewables
represented a majority. On the other hand, investment in coal and
gas-fired electricity generation was less than half the recorded
investment made in solar, wind and other renewables capacity.

The volatility and uncertainty of renewables are key characteristics of
the future power system. To study an optimal generation mix, we are
going to calculate the output of intermittent wind and solar energy by
simulation. I have already collected the wind and solar data for Alberta
over a period of ten years and used this data to calculated the
simulated output for typical wind and solar facilities located across
the province.

\subsection{Load Data}\label{load-data}

We collected Alberta's load data up to 2015 from AESO. The load duration
curve can then be constructed by arranging load during each hour through
the year from highest to lowest as indicated in Figure 1. The maximum
load in Alberta 2015 was 11,229 MW, and the minimum was 7,203 MW.
Further, this dataset includes observed output from various energy
sources. This information can be used as a benchmark to calibrate cost
functions. Figure 1 provides some indication of how the demand or load
pattern might change if must run solar energy that occurs during the day
is subtracted from load (see Figure 1).

{{[}CHART{]}}

Figure 1 Alberta Load Duration Curve MW 2015

Source: Author calculation, data from AESO

\subsection{Wind Data}\label{wind-data}

Wind speed data was collected from 17 locations in Alberta, and a
weighted average was taken across regions (weighted in favor of Pincher
Creek, which exhibited above-average wind speeds) for a period of ten
years, from 2006-2015 inclusive.

We simulate the wind power that could have been generated every hour for
the period 2006 through 2015 using wind-turbine power curves and the
data on wind speeds. Hourly wind power output is then subtracted from
demand to obtain the load that must be met by the various fossil-fuel
and other generating assets comprising the Alberta electricity system.

The Alberta electricity grid is characterized by industrial consumers
and three main types of generation -- coal, natural gas, and
co-generation. We collect the load data for 2015 and draw a load
duration curve for Alberta. The actual capacity and generation in
Alberta Electric System for 2014 are also used in the analysis.

Alberta has an enormous wind potential, but still relies mostly on
fossil fuels.\footnote{Source:
  http://canwea.ca/wind-facts/wind-facts-alberta/ . And new data is
  possible to get from NREL which also provides the energy output
  estimation. {[}accessed April 4, 2017{]}} Alberta is Canada's third
largest wind energy market with 1500 MW capacity. Wind power generates
9\% of the electricity in Alberta, while Alberta heavily relies on
fossil fuel like coal and gas for electricity generation.

The supply structure of the power system in Alberta is going to change
in the foreseeable future. Alberta has invested wind farms in recent
years, and is expected to increase wind capacity by thousands of MW over
next 15 years (Canadian Wind Energy Association 2016). Alberta already
planned to build more wind farms.\footnote{New wind projects data is
  available from
  https://www.aeso.ca/market/market-and-system-reporting/long-term-adequacy-metrics/
  {[}accessed April 4, 2017{]}} The land-based wind power has relatively
lower LCOE and, recently, the LCOE of solar photovoltaic has fallen
dramatically (Figure 3).

{{[}CHART{]}}

Figure 2 Annual Installed Wind Power Capacity (Megawatts)\\
Source: Author calculation and data from Canadian Wind Energy
Association (CanWEA).

\subsection{Solar Data}\label{solar-data}

With increasing capacity in renewable energy, in some countries
renewable energy has already begun to play an important role. In May,
2016, solar generated more electricity than coal in the UK, a new record
for May. The increase in solar-powered electricity comes as the amount
of coal power in the national grid fell to zero several times during
that month, which is thought to be the first time this had happened in
more than 100 years. Solar made up 6 percent of the UK's electricity in
May, while coal made up only 4 percent (Sheffield 2016). Furthermore, in
terms of the levelized cost of electricity (LCOE), renewables are
competitive with conventional energy at the utility level. Figure 3
shows that wind and solar energy have near grid parity, which means that
wind and solar power are cost competitive with fossil fuels at the
utility level. The dashed line is the upper and lower bound of 75\%
confidence interval of costs for coal and gas.