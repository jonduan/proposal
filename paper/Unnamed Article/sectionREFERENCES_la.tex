\section{REFERENCES }\label{references}

Ahmed, T. (2016). \emph{Modeling the renewable energy transition in
Canada: techno-economic assessments for energy management}. Springer.

Alberta Electric System Operator (AESO). (2010). Phase Two Wind
Integration. Working Paper, Market Services Division. Calgary, Alberta.
https://www.aeso.ca/downloads/Phase\_II\_Wind\_Integration\_Recommendation\_-\_Final.pdf.
{[}Accessed Apr 4, 2017.{]}

Bakke, G. A. (2016). \emph{The grid: the fraying wires between Americans
and our energy future}. Bloomsbury USA.

Buysse, J., Fernagut, B., Harmignie, O., de Frahan, B. H., Lauwers, L.,
Polome, P., Van Meensel, J. (2007). Farm-based modelling of the EU sugar
reform: impact on Belgian sugar beet suppliers. \emph{European Review of
Agricultural Economics}, \emph{34}(1), 21--52.

Biggar, D. R., Hesamzadeh, M. R. (2014). \emph{The Economics of
Electricity Markets}. \emph{The Economics of Electricity Markets}.

Behboodi, S., Chassin, D. P., Djilali, N., Crawford, C. (2017).
Interconnection-wide hour-ahead scheduling in the presence of
intermittent renewables and demand response: A surplus maximizing
approach.~\emph{Applied Energy},~189, 336-351.

Benitez, L.E., Benitez, P.C., van Kooten, G.C. (2008). The Economics of
Wind Power with Energy Storage, \emph{Energy Economics} 30(4):
1973-1989.

Castronuovo, E.D., Lopes, J.A.P., (2004). Optimal operation and hydro
storage sizing of a wind-hydro power plant. Electrical Power and Energy
Systems 26, 771--778.

Canadian Wind Energy Association. (2016). Wind energy continues rapid
growth in Canada in 2015. Retrieved from
http://canwea.ca/wind-energy-continues-rapid-growth-in-canada-in-2015/
{[}Accessed April 6, 2017{]}

Chen, X., \& Onal, H. (2012). Modeling Agricultural Supply Response
Using Mathematical Programming and Crop Mixes. \emph{American Journal of
Agricultural Economics}, \emph{94}(3), 674--686.

David Suzuki Foundation. (2016). Coal-fired power worsening health and
climate nation-wide \textbar{} News. Retrieved from
http://www.davidsuzuki.org/media/news/2016/11/coal-fired-power-worsening-health-and-climate-nation-wide/
{[}Accessed April 6, 2017{]}.

EIA, (2010). Updated Capital Cost Estimates for Electricity Generation
Plants. U.S. Energy Information Administration.
http://www.eia.gov/oiaf/beck\_plantcosts/index.html {[}Accessed April 5,
2016{]}.

Environment Canada. (2011). \emph{Canada's Emissions Trends}.
https://doi.org/EN81-18/2013E-PDF {[}Accessed April 5, 2016{]}.

Frew, B., Gallo, G., Brinkman, G., Milligan, M., Clark, K., Bloom, A.
(2016).~\emph{Impact of Market Behavior, Fleet Composition, and
Ancillary Services on Revenue Sufficiency~}(No. NREL/PR-6A20-66384).
NREL (National Renewable Energy Laboratory (NREL), Golden, CO (United
States)).

Government of Alberta, (2015). Climate Leadership Plan will protect
Albertans' health, environment and economy. November 22.
http://www.alberta.ca/release.cfm?xID=
38885E74F7B63-A62D-D1D2-E7BCF6A98D616C09 {[}accessed February 10,
2016{]}.

Government of Canada, (2011). Reduction of Carbon Dioxide Emissions from
Coal-Fired Generation of Electricity Regulations, \emph{Canada Gazette}
Vol. 145, No. 35.

Government of Canada, (2016). \emph{Canada's Second Biennial Report on
Climate Change}. Environment and Climate Change Canada. 52pp.
https://www.ec.gc.ca/GES-GHG/default.asp?lang=En\&n=02D095CB-1
{[}Accessed April 5, 2017{]}.

Guo, G. (2017). \emph{Why Is Asia Returning to Coal?} The Diplomat.
Retrieved from
http://thediplomat.com/2017/02/why-is-asia-returning-to-coal/
{[}Accessed 5 Apr. 2017{]}.

Hallberg, P., Claxton, A., Raedemaeker, P. De, Holm, A., Foosnaes,
Martin, J. G., Mallet, P. (2011). Views on Demand-Side Participation:
Involving Customers, Improving Markets, Enhancing Network Operation, 23.

Hanley, S. (2016). China, Japan, Russia, And South Korea Plan Super Grid
For Renewable Energy −. Retrieved from
http://solarlove.org/china-japan-russia-south-korea-super-grid-renewable/
{[}Accessed April 6, 2017{]}.

Heckelei, T. (2002). Calibration and estimation of programming models
for agricultural supply analysis, University of Bonn Habitation Thesis,
Bonn.

Heckelei, T. and Britz, W. (2000). Positive mathematical programming
with multiple data points, Cahiers d'Economie et Sociologie Rurales 57,
28--50.

Heckelei, T. and Britz, W. (2005). Models based on positive mathematical
programming: state of the art and further extensions, in Arfini, F.
(ed.), Modelling Agricultural Policies: State of the Art and New
Challenges. Proceedings of the 89th European Seminar of the European
Association of Agricultural Economics, University of Parma, Parma, pp.
48--73.

Heckelei, T. and Wolff, H. (2003). Estimation of constrained
optimisation models for argicutural supply analysis based on maximum
entropy, \emph{European Review of Agricultural Economics} 30, 27--50.

Heckelei, T., Britz, W., Zhang, Y. (2012). Positive Mathematical
Programming Approaches -- Recent Developments in Literature and Applied
Modelling. \emph{Bio-Based and Applied Economics}, \emph{1}(1),
109--124.

Henton, D., Varcoe, C. (2015). Early shutdown of coal-fired power plants
could cost billions of dollars:~analyst, \emph{Calgary Herald}, November
25.

Howitt, R. E. (1995). Positive Mathematical Programming. \emph{American
Journal of Agricultural Economics}, \emph{77}(2), 329.

International Energy Agency. (2014). Technology Roadmap Solar
Photovoltaic Energy - 2014 edition. Retrieved from
https://www.iea.org/publications/freepublications/publication/TechnologyRoadmapSolarPhotovoltaicEnergy\_2014edition.pdf
{[}Accessed April 6, 2017{]}.

Joskow, P.L., (2006). Competitive Electricity Markets and Investment in
New Generating Capacity. CEEPR WP 06-009. Center for Energy and
Environmental Policy Research, Department of Economics and Sloan Scholl
of Management, MIT, Cambridge, MA. April 28. 74pp.
http://dspace.mit.edu/bitstream/handle/1721.1/45055/2006-009.pdf?sequence=1
{[}accessed Apr. 4, 2017{]}.

Joskow, P. L. (2008). Capacity payments in imperfect electricity
markets: Need and design. \emph{Utilities Policy}, \emph{16}(3),
159--170.

Joskow, P.L., (2011). Comparing the Costs of Intermittent and
Dispatchable Generating Technologies, \emph{American Economic Review,
Papers and Proceedings} 101(3): 238-241.

Joskow, P. L. (2012). Creating a Smarter U.S. Electricity Grid.
\emph{The Journal of Economic Perspectives} , \emph{26}(1), 29--47.

Lazard. (2016). \emph{Levelized Cost of Storage - Volume 2}.
https://doi.org/10.1080/14693062.2006.9685626 {[}accessed Apr. 4,
2017{]}.

Lazard. (2016). \emph{Lazard's Levelized Cost of Energy Analysis}.
Retrieved from
https://www.lazard.com/media/438038/levelized-cost-of-energy-v100.pdf
{[}accessed Apr. 4, 2017{]}.

Liik, O., Oidram, R., Keel, M., (2003). Estimation of Real Emissions
Reduction caused by Wind Generators. Paper presented at the
International Energy Workshop, 24--26 June. IIASA, Laxenburg, Austria.

Love, M.L. Pitt, Niet, T., McLean, G., (2003). Utility-Scale Energy
Systems: Spatial and Storage Requirements, Working Paper. Institute for
Integrated Energy Systems. University of Victoria, BC.

Lovering, J.R., Yip, A., Nordhaus, T. (2016). Historical Construction
Costs of Global Nuclear Power Reactors, \emph{Energy Policy} 91:
371-382.

Lund, H., (2005). Large-scale integration of wind power into different
energy systems, Energy 30(13): 2402-2412.

Korpaas, M., Holen, A.T., Hildrum, R. (2003). Operation and sizing of
energy storage for wind power plants in a market system. Electrical
Power and Energy Systems 25, 599--606.

Seyboth, K, Sverrisson, F., Appavou, F., Brown, A., Epp, B., Leidreiter,
A., Sovacool, B. (2016). \emph{Renewables 2016 Global Status Report}.
\emph{Global Status Report} .

Maddaloni, J. D., Rowe, A.M., van Kooten, G.C. (2008). Network
constrained wind integration on Vancouver Island, Energy Policy 36(2):
591-602.

McCarthy, S. (2016). Ottawa to phase out coal, aims for virtual
elimination by 2030 - The Globe and Mail. Retrieved from
http://www.theglobeandmail.com/report-on-business/industry-news/energy-and-resources/ottawa-to-announce-coal-phase-out-aims-for-virtual-elimination-by-2030/article32953930/
{[}Accessed April 6, 2017{]}.

Milligan, M., Schwartz, M., Wan, Y., (2003). Statistical Wind Power
Forecasting Models: Results for U.S. Wind Farms. National Renewable
Energy Laboratory, Golden, CO.

Murphy, K., Knaus, C. (2017). South Australian blackout blamed on
thermal and wind generator failures, plus high demand \textbar{}
Australia news \textbar{} The Guardian. Retrieved from
https://www.theguardian.com/australia-news/2017/feb/15/south-australian-blackout-caused-by-demand-and-generator-failures-market-operator-says
{[}Accessed April 4, 2017{]}.

Nolan, J., Parker, D., Van Kooten, G. C., Berger, T. (2009). An overview
of computational modeling in agricultural and resource economics.
\emph{Canadian Journal of Agricultural Economics} , \emph{57}(4),
417--429.

Osborn, L. (n.d.). Sunniest Places in Canada - Current Results.
Retrieved from
https://www.currentresults.com/Weather-Extremes/Canada/sunniest-places.php
{[}Accessed April 6, 2017{]}.

Oswald, J., Raine, M., Ashraf-Ball, H. (2008). Will British weather
provide reliable electricity? Energy Policy 36(8), 3202--3215

Paris, Q., Howitt, R. E. (1998). An Analysis of Ill-Posed Production
Problems Using Maximum Entropy. \emph{American Journal of Agricultural
Economics}, \emph{80}(1), 124--138.

Paris, Q. (2015). PMP and Uniqueness of Calibrating Solution: A
Revision. \emph{SSRN Electronic Journal}.

Payton, M. (2016). Nearly 50 countries vow to use 100\% renewable energy
by 2050. from
http://www.independent.co.uk/news/world/renewable-energy-target-climate-united-nations-climate-change-vulnerable-nations-ethiopia-a7425411.html
{[}Accessed April 06, 2017{]}

Petsakos, A., Rozakis, S. (2015). Calibration of agricultural risk
programming models. \emph{European Journal of Operational Research},
\emph{242} (2), 536--545.

Pitt, L., van Kooten, G.C., Djihali, N. (2005). Utility-Scale Wind
Power: Impacts of Increased Penetration.

Prescott, R., van Kooten, G.C., Zhu, H. (2006). The Potential for Wind
Energy Meeting Electricity Needs on Vancouver Island.

Prescott, R., van Kooten, G.C. (2009). Economic costs of managing of an
electricity grid with increasing wind power penetration with increasing
wind power penetration. \emph{Climate Policy}, \emph{9}(2):155--168.

Ravindran, A., Reklaitis, G.V., Ragsdell, K.M. (2006)~\emph{Engineering
optimization: methods and applications}. John Wiley \& Sons.

Scorah, H., Sopinka, A., van Kooten, G. C. (2012). The economics of
storage, transmission and drought: Integrating variable wind power into
spatially separated electricity grids. \emph{Energy Economics},
\emph{34} (2), 536--541.

Scorah, H. (2010). Integration of Wind Power in Deregulated Power
Systems. \emph{Thesis}.

Sheffield, H. (2016). Solar just made more power than coal in the UK for
the first month ever \textbar{} The Independent. Retrieved from
http://www.independent.co.uk/news/business/news/solar-just-made-more-power-than-coal-in-the-uk-for-the-first-month-ever-a7074621.html
{[}Accessed April 6, 2017{]}.

Stanley, T. D., Doucouliagos, H. (2012). \emph{Meta-regression analysis
in economics and business}. Routledge.

Stoft, S., (2002). \emph{Power System Economics. Designing Markets for
Electricity}. Piscataway, NJ: IEEE Press/Wiley-InterScience.

Timilsina, G.R., van Kooten, G.C., Narbel, P.A., (2013). Global Wind
Power Development: Economics and Policies, \emph{Energy Policy} 61:
642-652.

U.S. Energy Information Administration, (2013). Form EIA-860, Annual
Electric Generator Report. Operable Generating Units in the United
States by State and Energy Source, 2013.
http://www.eia.gov/electricity/data/eia860/index.html {[}Accessed May 5,
2015{]}.

UNEP and Bloomberg, F. S. (2016). Global Trends in Renewable Energy
Investment. Retrieved from
http://fs-unep-centre.org/sites/default/files/publications/globaltrendsinrenewableenergyinvestment2016lowres\_0.pdf
{[}Accessed April 4, 2017{]}

van Kooten, G. C. (2010). Wind power: the economic impact of
intermittency. \emph{Letters in Spatial and Resource Sciences},
\emph{3} (1), 1--17.

van Kooten, G.C. (2012). Climate Change, Climate Science and Economics:
Prospects for a Renewable Energy Future. Dordrecht, NL: Springer.

van Kooten, G. C., Johnston, C., Wong, L. (2013). Wind versus nuclear
options for generating electricity in a carbon-constrained world:
strategizing in an energy-rich economy.~\emph{American Journal of
Agricultural Economics},~\emph{95}(2), 505-511.

Wang, H. (2016).~Economic Mechanisms for Integrating Smart Grid
Technologies into Power Systems~(Doctoral dissertation, The Chinese
University of Hong Kong (Hong Kong)).