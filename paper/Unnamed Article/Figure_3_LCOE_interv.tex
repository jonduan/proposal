Figure 3 LCOE interval in 2009 and 2014

Source: Author calculations and
http://en.openei.org/apps/TCDB/transparent\_cost\_database

Despite Alberta's high latitude (which usually does not bode well for
solar power potential), its sunny summers and cold but sunny winters are
conducive to solar cell performance (Eisenmenger 2011). The Canadian
Weather Energy and Engineering Datasets (CWEEDS) provide data on a range
of meteorological elements, recorded hourly, over the decade 1996-2005
for ten regions in Alberta (Environment Canada 2016). The data used to
estimate the potential power output of one 236W-solar module
(photovoltaic panel).

Alberta has abundant solar energy. Southern Alberta is part of the sun
belt of Canada. According to the measurement of Environment of Canada,
three top sunniest cities in Canada are from Alberta. These three
cities, Medicine Hat, Lethbridge, and Suffield enjoy about 2500 hours of
sun a year (Osborn n.d.). Compared to other regions with highly
developed solar power, Alberta has better solar power potential. For
example, Calgary has 1292~kWh/kW of solar potential, whereas Berlin
(Germany) 868~kWh/kW, and London (England) has 728~kWh/kW of solar
potential (Ahmed 2016).